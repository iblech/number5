\documentclass[]{article}
\usepackage[T1]{fontenc}
\usepackage[latin1]{inputenc}
\usepackage[german]{babel}
\usepackage{amsmath,amsthm,amssymb,amscd,color,graphicx}
  
\theoremstyle{remark}
\newtheorem{satz}{Satz}
\newtheorem{aufg}{Aufgabe}
\newtheorem{definition}{Definition}

%opening
\title{S�tze mit Fibonaccizahlen, dem goldenen Schnitt und der Zahl 5}
\author{}

\begin{document}

\maketitle

\begin{abstract}
Eine Sammlung h�bscher und teilweise �berraschender Ergebnisse, worin Fibonaccizahlen, der goldene Schnitt oder die Zahl 5 eine besondere Rolle spielen. Keine Beweise.
\end{abstract}

\section{Fibonaccizahlen}
\begin{definition}
Die Folge $(f_n)_{n\geq 0}$ der Fibonaccizahlen ist rekursiv gegeben durch:
\begin{equation*}
f_0 = 0,\quad f_1 = 1,\quad f_{n+2} = f_{n+1} + f_{n} 
\end{equation*}
\end{definition}

\begin{satz}
Die gew�hnliche und die exponentiell erzeugende Funktion der Fibonaccizahlen $G_f$ resp. $E_f$ sind:
\begin{align*}
G_f(x) :=& \sum_{n\geq 0} f_nx^n = \frac{x}{1-x-x^2}, \\
E_f(x) :=& \sum_{n\geq 0} \tfrac{1}{n!}f_n x^n = \tfrac{1}{\sqrt{5}}\exp\left(\tfrac{1+\sqrt{5}}{2}\, x\right) -
\tfrac{1}{\sqrt{5}}\exp\left(\tfrac{1-\sqrt{5}}{2}\, x\right)
\end{align*}
\end{satz}

\begin{satz}
Sei $B_n$ eine $n\times n$-Matrix der Form
\[
\begin{pmatrix}
1 & 1 & & &  &  \\
1 & \ddots & \ddots   & &\multicolumn{2}{c}{\text{\huge 0}}  \\
&\ddots &\ddots &\ddots  & &  \\
& &\ddots &\ddots &\ddots   &  \\
& & & \ddots &\ddots  &   1 \\
 \multicolumn{2}{c}{\text{\huge 0}}  & & &  1&  1\\
\end{pmatrix}
\] 
Dann ist die Permanente von $B_n$ gleich $ f_{n+1} $.
\end{satz}

\section{Goldener Schnitt}
\begin{definition}
Der goldene Schnitt $\Phi$ ist definiert als $\Phi := \tfrac{1+\sqrt{5}}{2}$. 
\end{definition}
\begin{satz}
Der goldene Schnitt erf�llt die Gleichungen
\[
\Phi = 1 + \frac{1}{\Phi}, \qquad \Phi^{n+1} = f_{n+1}\Phi + f_n.
\]
\end{satz}

\begin{satz}
Sei $f : \mathbb{R}^{\geq 0} \longrightarrow\mathbb{R}^{\geq 0} $ diejenige differenzierbare und bijektive Funktion, deren erste Ableitung gleich ihrer Umkehrfunktion ist. Dann ist $f$ gegeben durch:
\[
f(x) = \left(\tfrac{1}{\Phi}\right)^{\tfrac{1}{\Phi}} x^\Phi.
\]
\end{satz}

\section{Die Zahl 5}
\begin{definition}
Die Zahl $5$ ist gleich $1+1+1+1+1$.
\end{definition}
\begin{satz} (Pentagonalzahlensatz.) Die Pentagonalzahlen sind definiert durch $p_n := \tfrac{3n^2-n}{2}$ und berechnen die Anzahl der Steine, die ben�tigt werden, um $n$ ineinander liegende regelm��ige F�nfecke zu legen, welche eine Ecke gemeinsam haben.\\
Die Partitionszahlen werden bezeichnet durch $p(n)$ und z�hlen die M�glichkeiten, die Zahl $n$ als Summe positiver ganzer Zahlen zu schreiben. Dann gilt:
\begin{align*}
\frac{1}{(1-x)(1-x^2)(1-x^3)(1-x^4)\ldots} &= \sum_{n\geq 0} p(n) \,x^n\qquad \text{und} \\
(1-x)(1-x^2)(1-x^3)(1-x^4)\ldots &= \sum_{n \in \mathbb{Z}} (-1)^n x^{p_n}.
\end{align*}
\end{satz}
\begin{satz} (F�nferschritte bei Partitionszahlen.)
\[
\frac{5\left((1-x^5)(1-x^{10})(1-x^{15})\ldots \right)^5 }{\left((1-x)(1-x^2)(1-x^3)(1-x^4)\ldots\right)^6} 
= \sum_{n\geq 1} p(5n-1)\,x^n
\]
\end{satz}
\end{document}
