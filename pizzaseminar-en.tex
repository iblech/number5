\documentclass[a4paper,english,twoside]{scrartcl}

\usepackage[utf8]{inputenc}
\usepackage[english]{babel}
\usepackage{amsmath,amsthm,amssymb,amscd,color,graphicx,environ,mathtools}
\usepackage{framed}
\usepackage[protrusion=true,expansion=true]{microtype}
\usepackage{lmodern}
\usepackage{multicol}
\usepackage[normalem]{ulem}
\usepackage{hyperref}

\newcommand{\defeq}{\vcentcolon=}

\setlength{\unitlength}{1cm}

\setlength\parskip{\medskipamount}
\setlength\parindent{0pt}

\renewcommand*\theenumi{\alph{enumi}}
\renewcommand{\labelenumi}{\theenumi)}

\newlength{\aufgabenskip}
\setlength{\aufgabenskip}{1.4em}
\newcounter{aufgabennummer}
\newenvironment{aufgabe}[1]{
  \refstepcounter{aufgabennummer}
  \textbf{Exercise \theaufgabennummer.} \emph{#1} \par
}{\vspace{\aufgabenskip}}
\newenvironment{aufgabe*}[1]{
  \refstepcounter{aufgabennummer}
  \textbf{Exercise* \theaufgabennummer.} \emph{#1} \par
}{\vspace{\aufgabenskip}}
\newenvironment{aufgabeE}[1]{\begin{aufgabe}{#1}\begin{enumerate}}{\end{enumerate}\end{aufgabe}}

\clubpenalty=10000
\widowpenalty=10000
\displaywidowpenalty=10000

\newcommand{\NN}{\mathbb{N}}
\newcommand{\RR}{\mathbb{R}}
\DeclarePairedDelimiter{\floor}{\lfloor}{\rfloor}

\begin{document}

\thispagestyle{empty}
Institut für Mathematik \\
Universität Augsburg \\
Ingo Blechschmidt

\begin{center}
  \textbf{Pizza seminar in mathematics} \\
  \emph{The secret of the number 5}
\end{center}
\vspace{0.5em}

\begin{center}\href{https://commons.wikimedia.org/wiki/File:Roof_hafez_tomb.jpg}{\includegraphics[width=0.60\textwidth]{hafez-tomb}} \\
\scriptsize Tomb of Hafez (Iran); photo by Pentocelo (CC-licensed)\par
\end{center}
\vspace{0.5em}

The \emph{sequence of Fibonacci numbers} begins with
\[
  f_0 \defeq 1, \quad
  f_1 \defeq 1, \quad
  f_2 \defeq 2, \quad
  f_3 \defeq 3, \quad
  f_4 \defeq 5, \quad
  f_5 \defeq 8.
\]
So for~$n \geq 1$ it holds that $f_{n+1} = f_n + f_{n-1}$. The \emph{golden ratio}
is the number~$\Phi \defeq (1+\sqrt{5})/2$.
\vspace{1em}
\enlargethispage{0.5em}

\begin{aufgabe}{One box disappears!}
The two figures below are obviously made up from the same pieces. But they
have different area!
\begin{enumerate}
\item What happened?
\item What's the relation with approximation by truncated continued fractions?
Which number (which is very much related to the golden ratio) is relevant?
\end{enumerate}
\begin{center}
  \includegraphics[scale=0.25]{ein-kaestchen-verschwindet}
\end{center}
\end{aufgabe}

\begin{aufgabe}{Squares of Fibonacci numbers}
There is the identity~$f_0^2 + f_1^2 + \cdots + f_n^2 = f_n f_{n+1}$.
\begin{enumerate}
\item Proof this by induction.
\item Argue that the following sketch also proves the assertion.
\end{enumerate}
\begin{center}
  \includegraphics[scale=0.3]{fibonacci-quadrate}
\end{center}
\end{aufgabe}

\begin{aufgabe}{About Fibonacci numbers and rectangles}
Let a row of~$n$ squares be given, forming a~$(1 \times n)$-rectangle. We want
to cover this rectangle by minominos, that is~$(1 \times 1)$-rectangles, and by
dominos, that is~$(1 \times 2)$-rectangles.

Show that the number of possible such coverings is~$f_n$.
\end{aufgabe}
% \url{https://www.math.hmc.edu/~benjamin/papers/DIE.pdf}

\begin{aufgabe}{A very naive algorithm for computing Fibonacci numbers}
A very naive method for calculating the~$n$-th Fibonacci number is to calculate
its two predecessors and then sum them. How many computational steps are
necessary if you \emph{don't} store intermediate results and therefore
recalculate lots of numbers? Why is the result of this exercise funny?
\end{aufgabe}

\begin{aufgabe}{Fibonacci-Zahlen im Pascalschen Dreieck}
The Fibonacci numbers show up in Pascal's triangle. Prove that.
\begin{center}
\includegraphics[scale=0.3]{pascal-fibonacci}
\end{center}
\end{aufgabe}

\begin{aufgabe}{A lemma about golden triangles}
\begin{enumerate}
\item Show: In an isosceles triangle with internal angles~$72^\circ$,
$36^\circ$, and~$72^\circ$ the two long sides divide the base in the golden
ratio.
\item Prove the following pentagon-decagon-hexagon identity: Let a regular
pentagon, a regular decagon, and a regular hexagon be inscribed in a circle.
Then for the side lengths~$P$,~$D$, and~$H$ the identity~$P^2 = D^2 + H^2$
holds.
\end{enumerate}
\end{aufgabe}

\begin{aufgabe}{Icosahedron and dodecahedron}
Let a regular dodecahedron be inscribed in a regular icosahedron (midpoints of
faces on vertices). Show that the ratio of the edge lengths of those two
objects is~$3 : \Phi$.
\end{aufgabe}

\begin{aufgabe}{Continued fraction expansion of the golden ratio}
\begin{enumerate}
\item Show: $\Phi = 1 + \frac{1}{\Phi}$. You earn valuable bonus points if you
manage to verify this identity by using the geometric definition of the golden
ratio: the total length of the segment is to the larger subsegment as the
larger subsegment is to the smaller subsegment.
\item Conclude:
\[ \Phi = 1 + \frac{1}{1 + \dfrac{1}{1 + \frac{1}{\ddots}}}. \]
\item Prove: The sucessive approximations to the golden ratio obtained by
truncating the continued fraction expansion are of the form~$f_{n+1}/f_n$.
\end{enumerate}
\end{aufgabe}

\begin{aufgabe}{Derivative vs. inverse}
Find a bijective and differentiable function~$\RR^+ \to \RR^+$ whose derivative
equals its inverse.

\emph{Tipp:} Have a go with the ansatz~$x \mapsto x^a$.
\end{aufgabe}

\begin{aufgabe}{Examples for continued fraction expansions}
\begin{enumerate}
\item Verify that~$\sqrt{2} = [1; 2, 2, \ldots]$.
\item Verify that~$\sqrt{3} = [1; 1, 2, 1, 2, 1, 2, \ldots]$.
\item What's the continued fraction expansion of~$\sqrt{5}$?
\item Convince yourself of the triviality of the following statement: Any real
number is the solution of some quadratic equation.
\item Prove: Every number whose continued fraction expansion is periodic is
solution of some quadratic equation \emph{with rational coefficients}.
\end{enumerate}
\end{aufgabe}

\begin{aufgabe}{Continued fraction expansion and the Euclidean algorithm}
Let~$x$ be a nonnegative real number. Assume that the Euclidean algorithm
produces the following equations:
\begin{align*}
  x &= a_0 \cdot \sbox0{$r_0$}\makebox[\wd0][l]{$1$} + r_0 \\
  1 &= a_1 \cdot r_0 + r_1 \\
  r_0 &= a_2 \cdot r_1 + r_2 \\
  r_1 &= a_3 \cdot r_2 + r_3 \\
  r_2 &= a_4 \cdot r_3 + r_4
\end{align*}
And so on. The numbers~$a_n$ are nonnegative integers and the residues~$r_n$
are smaller than the second factor of the respective adjacent product. Prove:
\[ x = a_0 + \dfrac{1}{a_1 + \dfrac{1}{a_2 + \dfrac{1}{\ddots}}}. \]
\end{aufgabe}
% muss vor nächster Aufgabe stehen

% muss nach vorheriger Aufgabe stehen
\enlargethispage{1.8em}
\begin{aufgabe}{The discovery of irrationality}
Als Entdecker der Irrationalität gilt der griechische Mathematiker Hippasos von
Metapont. Er erkannte, dass der \emph{goldene Schnitt} irrational ist. Damit
erschütterte er die Schule der Pythagoreer, denn diese waren von dem Kredo
\emph{Alles ist Zahl} überzeugt, wobei sie mit "`Zahl"' \emph{rationale Zahl}
meinten. Ironischerweise kam der goldene Schnitt auch noch im Erkennungszeichen
der Pythagoreer vor, dem Pentagramm. In dieser Aufgabe möchten wir uns auf
Metaponts Spuren begeben.
\begin{enumerate}
\item Sei~$x$ eine rationale Zahl. Zeige, dass die Anwendung des euklidischen
Algorithmus auf~$x$ (wie in der vorherigen Aufgabe beschrieben)
\emph{terminiert}, das heißt nach einer gewissen endlichen Anzahl von Schritten
den Rest Null liefert.
\item Folgere: Die Kettenbruchentwicklung einer rationalen Zahl ist stets
endlich.
\item Verwende folgende Figur, um rein geometrisch einzusehen, dass der
euklidische Algorithmus beim goldenen Schnitt als Startwert \emph{nicht}
terminiert. Somit ist der goldene Schnitt irrational.
\item Halte die Augen offen, wann das Buch \emph{Sternstunden der Mathematik}
von Jost-Hinrich Eschenburg erscheint, das diese Station der Geschichte und
viele weitere genauer beleuchtet. Der Vorentwurf steht im Digicampus. Es
wird ein rundum schönes Buch!
\end{enumerate}
\begin{center}
\href{http://classicalastrologer.me/2013/01/13/venus-the-golden-mean-the-pentagram/}{\includegraphics[width=0.4\textwidth]{pentagramm-irrational}}
\end{center}
\end{aufgabe}

\begin{aufgabe}{Zahlentheoretische Irrationalitätsbeweise}
\begin{enumerate}
\item Zeige, dass~$\sqrt{2}$ irrational ist, in dem du die Annahme, es gäbe
ganze Zahlen~$a$ und~$b$ mit~$\sqrt{2} = a/b$, zu einem Widerspruch führst.
Bonuspunkte gibt es, wenn du in deinem Beweis nicht ohne Beschränkung der
Allgemeinheit voraussetzen musst, dass der Bruch~$a/b$ vollständig gekürzt ist.
\item Zeige nach demselben Muster, dass~$\Phi$ irrational ist. Nutze die
Identität~$\Phi = 1 + 1/\Phi$.
\end{enumerate}
\end{aufgabe}

\begin{aufgabe*}{Rekursionsformel für die Kettenbruchapproximationen}\label{rec}
Sei ein unendlicher Kettenbruch der Form
\[ [c_0;c_1,c_2,\ldots] = c_0 + \dfrac{1}{c_1 + \dfrac{1}{c_2 + \dfrac{1}{\ddots}}} \]
gegeben. Wenn man diesen sukzessive abschneidet, erhält man die Approximationen
\[ c_0, \qquad
  c_0 + \dfrac{1}{c_1} = \frac{c_0c_1 + 1}{c_1}, \qquad
  c_0 + \dfrac{1}{c_1 + \dfrac{1}{c_2}} = \frac{c_0c_1c_2 + c_0 +
  c_2}{c_1c_2+1}
\]
und so weiter. Wir definieren rekursiv Folgen~$A_{-1},A_0,\ldots$
und~$B_{-1},B_0,\ldots$:
\begin{align*}
  A_{-1} &= 1 & B_{-1} &= 0 \\
  A_0 &= c_0 & B_0 &= 1 \\
  A_{n+1} &= c_{n+1} A_n + A_{n-1} & B_{n+1} &= c_{n+1} B_n + B_{n-1}
\end{align*}
Zeige für alle~$n \geq 0$: Der Bruch, der sich durch Abschneidung nach
Stelle~$n$ ergibt, ist~$A_n/B_n$.

\emph{Tipp:} Versuche einen Induktionsbeweis. Verwende im Induktionsschritt die
zentrale Einsicht, dass der Kettenbruch~$[c_0;c_1,\ldots,c_n,c_{n+1}]$ gleich
dem um ein Glied kürzeren Kettenbruch~$[c_0;c_1,\ldots,c_n+1/c_{n+1}]$ ist.
\end{aufgabe*}

\begin{aufgabe}{Unkürzbarkeit der Kettenbruchapproximationen}
Wie in Aufgabe~\ref{rec} schreiben wir~"`$A_n/B_n$"' für den Bruch, der sich
ergibt, wenn man die Kettenbruchentwicklung einer gegebenen Zahl abschneidet.
In dieser Aufgabe möchten wir zeigen, dass dieser Bruch stets schon gekürzt
ist.
\begin{enumerate}
\item Seien~$a$ und~$b$ zwei ganze Zahlen. Wieso sind~$a$ und~$b$ zueinander
teilerfremd, wenn es weitere ganze Zahlen~$p$ und~$q$ mit~$1 = pa + qb$ gibt?
\item Zeige für alle~$n \geq 0$: $A_{n+1} B_n - B_{n+1} A_n = (-1)^n$.
\item Ziehe das Fazit.
\end{enumerate}
\end{aufgabe}

\begin{aufgabe*}{Conways Armee}
Ein unendlich ausgedehntes Damebrett sei in zwei Hälften zerteilt. Im unteren
Teil darf man beliebig viele Damesteine platzieren. Ziel des Spiels ist es,
einen Damestein möglichst hoch in das obere Spielfeld zu
bringen. Dabei darf nur folgender Spielzug angewendet werden: Ein Stein
darf einen (horizontal oder vertikal) benachbarten Stein überspringen, wenn das
Zielfeld unbesetzt ist. Der übersprungene Stein wird dann aus dem Spiel
entfernt.
\begin{enumerate}
\item Überzeuge dich davon, dass man, um Höhe~1, 2, 3 bzw. 4 über der
Trennlinie zu erreichen, mit~2, 4, 8 bzw. \sout{16} 20 Steinen beginnen muss.
\item Zeige, dass Höhe~5 mit keiner endlichen Anzahl von Steinen erreichbar
ist.

\emph{Tipp:} Hier muss man auf geeignete Art und Weise eine von der
Feldbesetzung abhängige Größe definieren, die bei jedem Zug abnimmt. Man könnte
diese Größe zum Beispiel \emph{Energie} nennen. Dann kann man nachrechnen: Die
Energie von beliebig vielen Spielsteinen in der unteren Bretthälfte ist kleiner
als die Energie von auch nur einem einzigen Stein in Höhe~5. Eine mögliche
Definition für die Energie, die diese Anforderungen erfüllt, besteht darin, für
jeden vorhandenen Spielstein die Zahl~$(1/\Phi)^d$ aufzusummieren, wobei~$d$
die \emph{Manhattan-Entfernung} des Spielsteins zu einem beliebig ausgemachten
Ursprungsstein ist. Details findest du im Internet.
\end{enumerate}
\end{aufgabe*}

\begin{aufgabe*}{Verschoben aufsummierte Fibonacci-Zahlen}
In dieser Aufgabe betrachten wir die "`verschoben aufgeschriebenen"'
Fibonacci-Zahlen:
\begin{align*}
&0{,}1 \\
&0{,}01 \\
&0{,}002 \\
&0{,}0003 \\
&0{,}00005 \\
&0{,}000008 \\
&0{,}0000013
\end{align*}
Setzt man dieses Muster fort und summiert über alle Zeilen, so erhält man als
Summe exakt den Wert~$10/89$. Wieso?

\emph{Tipp:} Informierere dich über \emph{erzeugende Funktionen}, zum Beispiel
in dem tollen Buch \emph{Generatingfunctionology} von Herbert Wilf (auf der
Seite des Autors zu finden).
\end{aufgabe*}

\begin{aufgabe*}{Eine Zahl mit besonderer Dezimalbruchentwicklung}
Es gilt:
\[ \frac{1}{998999} =
  0{,}000\,001\,001\,002\,003\,005\,008\,013\,021\ldots. \]
\begin{enumerate}
\item Was ist daran besonders?
\item Inwiefern setzt sich das Muster auch nach~$987$ fort?
\item Erkläre das Phänomen.
\end{enumerate}
\end{aufgabe*}

\begin{aufgabe}{Spiel und Spaß mit den 10-adischen Zahlen}
Bei den gewöhnlichen reellen Zahlen stehen in ihrer Dezimalschreibweise vor dem
Komma nur endlich viele Ziffern, hinter dem Komma aber gelegentlich unendlich
viele Ziffern. Bei den~$10$-adischen Zahlen ist es genau umgekehrt: Vor dem
Komma dürfen unendlich viele Ziffern stehen, hinter dem Komma dagegen nur
endlich viele. Die Rechenverfahren zur Addition, Subtraktion und
Multiplikation, wie man sie aus der Schule kennt, funktionieren weitestgehend
unverändert. Die Division wird etwas komplizierter.
\begin{enumerate}
\item Vollziehe folgende Rechnung nach:
$\ldots 13562 + \ldots 90081 = \ldots 03643$.
\item Was ist $\ldots 99999 + 1$? Dabei ist~$1 = \ldots 00001$.
\item Was ist~$-123$?
\item Finde eine~$10$-adische Zahl~$x$ -- weder Null noch Eins -- mit~$x^2 = x$.
\item Überlege dir (oder schlage nach), wie in den 10-adischen Zahlen die
Division funktioniert. Die Division durch 3, 7, 9 und alle weiteren zu~10
teilerfremden ganzen Zahlen geht übrigens \emph{stets ohne Komma auf}.
\item Gibt es in den~$10$-adischen Zahlen eine Zahl~$x$ mit der
Eigenschaft~$x = 1 + 1/x$, oder äquivalent~$x^2 = x + 1$?
\item Wie sieht es in den~$11$-adischen Zahlen aus? (Schwer ohne umfangreiche
Tipps.)
\end{enumerate}
{\scriptsize
\emph{Bemerkung:} Die Gleichung in Teilaufgabe~d) kann man zu~$x \cdot (x-1) =
0$ umstellen. In den~$10$-adischen Zahlen kann also ein Produkt Null sein, ohne
dass einer der Faktoren Null ist. Wegen dieser schlechten Eigenschaft werden
die~$10$-adischen Zahlen kaum verwendet. \emph{Allerdings:} Verwendet man als
Basis nicht~$10$, sondern eine Primzahl, so gibt es dieses Problem nicht.
Die~$2$-adischen Zahlen werden gelegentlich in der Informatik und
die~$p$-adischen Zahlen, wobei~$p$ irgendeine Primzahl ist, überall in der
Zahlentheorie verwendet. Dort gibt es beispielsweise folgendes mächtiges
"`lokal-zu-global"' Prinzip: Eine Gleichung einer bestimmten Art hat genau dann
eine Lösung in~$\mathbb{Z}$, wenn sie eine Lösung in~$\mathbb{R}$ und jeweils
eine Lösung in allen~$p$-adischen Zahlen hat.\par}
\end{aufgabe}

\begin{aufgabe}{Pi auswendig lernen}
Die Nachkommaziffern von Pi sind nichts Kanonisches: Sie hängen von der
unkanonischen Wahl der Basis Zehn unseres Stellenwertsystems ab. Besser ist,
die Kettenbruchentwicklung von~Pi auswendig zu lernen. Sofern man die
Koeffizienten als Abstrakta begreift, ist das kanonisch. Die ersten
Koeffizienten lauten:

\scriptsize
3 7 15 1 292 1 1 1 2 1 3 1 14 2 1 1 2 2 2 2 1 84 2 1
1 15 3 13 1 4 2 6 6 99 1 2 2 6 3 5 1 1 6 8 1 7 1 2
3 7 1 2 1 1 12 1 1 1 3 1 1 8 1 1 2 1 6 1 1 5 2 2 3
1 2 4 4 16 1 161 45 1 22 1 2 2 1 4 1 2 24 1 2 1 3 1
2 1 1 10 2 5 4 1 2 2 8 1 5 2 2 26 1 4 1 1 8 2 42 2
1 7 3 3 1 1 7 2 4 9 7 2 3 1 57 1 18 1 9 19 1 2 18 1
3 7 30 1 1 1 3 3 3 1 2 8 1 1 2 1 15 1 2 13 1 2 1 4
1 12 1 1 3 3 28 1 10 3 2 20 1 1 1 1 4 1 1 1 5 3 2 1
6 1 4 1 120 2 1 1 3 1 23 1 15 1 3 7 1 16 1 2 1 21 2
1 1 2 9 1 6 4 127 14 5 1 3 13 7 9 1 1 1 1 1 5 4 1 1
3 1 1 29 3 1 1 2 2 1 3 1 1 1 3 1 1 10 3 1 3 1 2 1
12 1 4 1 1 1 1 7 1 1 2 1 11\par
% http://www.plouffe.fr/simon/constants/CFPiTerms20aa.txt

\end{aufgabe}

\end{document}

Eine ganz tolle Quelle:
http://www.math.brown.edu/~jhs/frintonlinechapters.pdf

Auch gut:
http://www.math.jacobs-university.de/timorin/PM/continued_fractions.pdf
http://www.millersville.edu/~bikenaga/number-theory/approximation-by-rationals/approximation-by-rationals.html
